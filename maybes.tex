Folgend liegt der Fokus hauptsächlich auf Möglichkeiten der Stressreduktion und -prävention mithilfe von Entspannungsmethoden wie Meditation und Achtsamkeitstraining.
Die sog. \textit{Mindfulness Based Stress Reduction} (folgend MBSR abgekürzt) wurde vor rund 40 Jahren von dem Molekularbiologen Dr. Jon Kabat-Zinn entwickelt und bezeichnet ein ein achtwöchiges Trainingsprogamm, welches Übungen aus dem Yoga, sowie Übungen zur eigenen Körperwahrnehmung sowie Aufmerksamkeit umfasst. Auf dieser Grundlage bauen mittlerweile verschiedene Therapieansätze auf \zit{MBSR1}. Verschiedene Studien belegen die Wirksamkeit bzw. positiven Auswirkungen solcher Praktiken im Umgang mit Stressbelastung, Ängsten und Depressionen.




Mittlerweile erkennen auch viele Arbeitgeber bzw. Unternehmen, dass entspanntere Mitarbeiter meißt gleichermaßen produktivere Mitarbeiter sind und bieten deshalb unter anderem Meditations- und Achtsamskeitkurse für Mananger oder teils gesamte Belegschaften an \zit{Handelsblatt}. Das Softwareunternehmen Adobe beispielsweise unterstützt seine Mitarbeiter mit verschiedenen Hilfestellungen und vergibt kostenfreie Abonnements für die Meditations- und Entspannungs-App \textit{Headspace} \zit{Adobe}. 
Aber auch außerhalb der Arbeitswelt entdecken immer mehr Menschen Apps für sich, die schnelle Entspannung für die breite Masse versprechen. So haben sich die Ausgaben für mobile Gesundheits- und Meditations-Apps zwischen 2016 und 2018 global rund verdreifacht \zit{SteigendeNutzung}.

metakompetenz --> Achtsamkeit



 Die Leitfadeninterviews, die mit den Studierenden geführt wurden, zeigten, dass \grqq Mindfulness Apps\grqq\ durchaus sehr gut zur Entspannung und Stressminderung beitragen können. Die Befragten, die bereits Erfahrungen mit derartigen Anwendungen gesammelt haben (über die Hälfte aller Befragten), sagten unter anderem aus, dass sie durch die Nutzung in akuten Stresssituationen schneller wieder entspannen und klare Gedanken fassen konnten.
 
 Neben der Bestätigung, dass derartige Anwendungen in gewissem Umfang positive Auswirkungen auf den Umgang mit Stress haben können, konnten auch ausschlaggebende Punkte erörtert werden, welche eine (regelmäßige) Nutzung einer solchen App begünstigen.  XXX