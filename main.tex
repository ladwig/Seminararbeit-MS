% Dokumentenklasse und Seitenränder
\documentclass[10pt]{article}
\usepackage[a4paper, total={6in, 8in}]{geometry}

% Font-Einstellungen: Arial
\usepackage{fontspec}
\setmainfont[
BoldFont=arialbd.ttf,
ItalicFont=ariali.ttf,
BoldItalicFont=arialbi.ttf
]{arial.ttf}

% First Indent und CSQuotes
\usepackage{indentfirst}
\usepackage{csquotes}

% \maketitle überschreiben für linksbündigen Text
\makeatletter
\renewcommand{\maketitle}{\bgroup\setlength{\parindent}{0pt}
\begin{flushleft}
  \Large{\textbf{\@title}}
\end{flushleft}\egroup
}
\makeatother

% \section überschreiben für kleinere Schriftgröße
\usepackage{titlesec}
\titleformat*{\section}{\normalsize\bfseries}

% Packages für Referenzen...
\usepackage[backend=biber,style=authoryear,citestyle=authoryear]{biblatex}
\setlength\bibitemsep{1.0\itemsep}
\addbibresource{references.bib}

% Deutsche Referenzen
\usepackage[ngerman]{babel}

% Zeilenabstand
\usepackage{setspace}
\onehalfspacing

% Abstand zwischen Aufzählungszeichen
\usepackage{enumitem}

% Globale Einrückungen je Absatz unterbinden
\setlength{\parindent}{0pt}

% Zeilenabstand bei Absätzen
\setlength{\parskip}{2pt}


% Makro von StackExchange für sekundäre Quellen
\DeclareCiteCommand{\secondaryciteauthor}{}{\printtext[bibhyperref]{\printnames{labelname}}}{}{}
\DeclareCiteCommand{\secondaryciteyear}{}{\bibhyperref{\printdate}}{\multicitedelim}{}
\newcommand{\szit}[2]{(\secondaryciteauthor{#1} \secondaryciteyear{#1}, zitiert nach \secondaryciteauthor{#2} \secondaryciteyear{#2})}

% Makro für Klammern um Zitierung
\newcommand{\zit}[1]{(\cite{#1})}

% ======================
% ANFANG DOKUMENT
% ======================

\title{Wie können "Mindfulness" Smartphone-Apps zur Stressbewältigung bei Studenten beitragen? }

\begin{document}

\maketitle

\begin{flushleft}
Daniel Ladwig \\* 
Hochschule für angewandte Wissenschaften Würzburg-Schweinfurt
\end{flushleft}



\section{EINLEITUNG}

Rund jeder vierte Studierende fühlt sich im Studienalltag ziemlich bis häufig gestresst \zit{gesundheitStudis2017}. Im Bundesdurchschnitt fühlen sich sogar mehr als 75 \% der befragten 18-29 Jährigen zunehmend immer gestresster. Als Auslöser, sog. Stressoren, werden hauptsächlich Beruf, Ausbildung bzw. Studium und soziale Verpflichtungen wie Termine und Freizeitaktivitäten genannt \zit{tkEntspannDich2016}. 

Doch was genau ist unter dem Begriff Stress zu verstehen? Eine einheitliche Definition für diese Begrifflichkeit ist sowohl im alltäglichen, als auch wissenschaftlichen Kontext nur schwer zu treffen, da mit Stress nicht nur ein auslösender Faktor, sondern oftmals auch die Reaktion oder Folge auf jenen Faktor gemeint ist \zit{StressAllgemein}. War Stress früher noch eine überlebenswichtige Funktion des Menschen um in Bedrohungssituationen schnell reagieren zu können, ist er heute oft Auslöser für Krankheiten wie beispielsweise Verspannungen, Schlaganfälle und psychosomatische Erkrankungen. 

Doch Stress ist nicht gleich Stress. Kurzfristig anhaltender Stress mit darauf folgenden Ruhephasen, sog. Eustress, wird als positiv und leistungssteigernd empfunden und ermöglicht es dem Menschen neue oder schwierige Situationen gut zu meistern. 
Problematischer für die psychische und physische Gesundheit des Menschen ist der Distress bzw. chronische Stress, also Stress der dauerhaft anhaltend (d.h. auch ohne entsprechende Entspannungsphasen) und als negativ empfunden wird, wobei das Stressempfinden des einzelnen immer individuell ist \zit{StressGrundwissen} \zit{ChronischerStress}. 

Um jenen \grqq ungesunden\grqq  Stress besser bewältigen zu können sind individuelle Erholungsphasen notwendig. Diese Phasen dienen nicht nur der Erholung  und Distanzierung von angespannten Situationen selbst, sondern auch als Präventivmaßnahme. Ist der Mensch grundsätzlich weniger angespannt, so hat er mehr Ressourcen zur Verfügung, die ihm bei der Bewältigung von stressigen Situationen helfen können. 
Beispiele für erholsame und regenerative Aktivitäten sind:
\begin{itemize}[itemsep=0.5mm, parsep=0pt]
\item  Sport
\item  Hand- und Heimwerken
\item  Spielen
\item Entspannungsübungen
\end{itemize}

\zit{Stressbewältigung}.
Folgend wird hauptsächlich auf Möglichkeiten der Stressreduktion und -prävention mithilfe von Entspannungsmethoden wie Meditation eingangen. 
Mittlerweile erkennen auch viele Arbeitgeber bzw. Unternehmen, dass entspanntere Mitarbeiter meißt gleichermaßen produktivere Mitarbeiter sind und bieten deshalb unter anderem Meditations- und Achtsamskeitkurse für Mananger oder teils gesamte Belegschaften an \zit{Handelsblatt}. Das Softwareunternehmen Adobe beispielsweise unterstützt seine Mitarbeiter mit verschiedenen Hilfestellungen und vergibt kostenfreie Abonnements für die Meditations- und Entspannungs-App \textit{Headspace} \zit{Adobe}. 
Aber auch außerhalb der Arbeitswelt entdecken immer mehr Menschen Apps für sich, die schnelle Entspannung für die breite Masse versprechen. So haben sich die Ausgaben für mobile Gesundheits- und Meditations-Apps zwischen 2016 und 2018 global rund verdreifacht \zit{SteigendeNutzung}.





    

\bigbreak


\section{Haupteil}


\nocite{*}
\printbibliography

\end{document}