% Dokumentenklasse und Seitenränder
\documentclass[10pt]{article}
\usepackage[a4paper, total={6in, 8in}]{geometry}

% Font-Einstellungen: Arial
\usepackage{fontspec}
\setmainfont[
BoldFont=arialbd.ttf,
ItalicFont=ariali.ttf,
BoldItalicFont=arialbi.ttf
]{arial.ttf}

% First Indent und CSQuotes
\usepackage{indentfirst}
\usepackage{csquotes}

% \maketitle überschreiben für linksbündigen Text
\makeatletter
\renewcommand{\maketitle}{\bgroup\setlength{\parindent}{0pt}
\begin{flushleft}
  \Large{\textbf{\@title}}
\end{flushleft}\egroup
}
\makeatother

% \section überschreiben für kleinere Schriftgröße
\usepackage{titlesec}
\titleformat*{\section}{\normalsize\bfseries}

% Packages für Referenzen...
\usepackage[backend=biber,style=authoryear,citestyle=authoryear]{biblatex}
\setlength\bibitemsep{1.0\itemsep}
\addbibresource{references.bib}

% Deutsche Referenzen
\usepackage[ngerman]{babel}

% Zeilenabstand
\usepackage{setspace}
\onehalfspacing

% Abstand zwischen Aufzählungszeichen
\usepackage{enumitem}

% Globale Einrückungen je Absatz unterbinden
\setlength{\parindent}{0pt}

% Zeilenabstand bei Absätzen
\setlength{\parskip}{2pt}


% Makro von StackExchange für sekundäre Quellen
\DeclareCiteCommand{\secondaryciteauthor}{}{\printtext[bibhyperref]{\printnames{labelname}}}{}{}
\DeclareCiteCommand{\secondaryciteyear}{}{\bibhyperref{\printdate}}{\multicitedelim}{}
\newcommand{\szit}[2]{(\secondaryciteauthor{#1} \secondaryciteyear{#1}, zitiert nach \secondaryciteauthor{#2} \secondaryciteyear{#2})}

% Makro für Klammern um Zitierung
\newcommand{\zit}[1]{(\cite{#1})}

% ======================
% ANFANG DOKUMENT
% ======================

\title{Wie können "Mindfulness" Smartphone-Apps zur Stressbewältigung bei Studenten beitragen? }

\begin{document}

\maketitle

\begin{flushleft}
Daniel Ladwig  \\* 
5017042 \\* 
Hochschule für angewandte Wissenschaften Würzburg-Schweinfurt
\end{flushleft}

\section{EINLEITUNG}

Rund jeder vierte Studierende fühlt sich im Studienalltag ziemlich bis häufig gestresst \zit{gesundheitStudis2017}. Im Bundesdurchschnitt fühlen sich sogar mehr als 75 \% der befragten 18-29 Jährigen zunehmend immer gestresster. Als Auslöser, sog. Stressoren, werden hauptsächlich Beruf, Ausbildung bzw. Studium und soziale Verpflichtungen wie Termine und Freizeitaktivitäten genannt \zit{tkEntspannDich2016}. 

Doch was genau ist unter dem Begriff Stress zu verstehen? Eine einheitliche Definition für diese Begrifflichkeit ist sowohl im alltäglichen, als auch wissenschaftlichen Kontext nur schwer zu treffen, da mit Stress nicht nur ein auslösender Faktor, sondern oftmals auch die Reaktion oder Folge auf jenen Faktor gemeint ist \zit{StressAllgemein}. War Stress früher noch eine überlebenswichtige Funktion des Menschen um in Bedrohungssituationen schnell reagieren zu können, ist er heute oft Auslöser für Krankheiten wie beispielsweise Verspannungen, Schlaganfälle und psychosomatische Erkrankungen. 

Doch Stress ist nicht gleich Stress. Kurzfristig anhaltender Stress mit darauf folgenden Ruhephasen, sog. Eustress, wird als positiv und leistungssteigernd empfunden und ermöglicht es dem Menschen neue oder schwierige Situationen gut zu meistern. 
Problematischer für die psychische und physische Gesundheit des Menschen ist der Distress bzw. chronische Stress, also Stress der dauerhaft anhaltend (d.h. auch ohne entsprechende Entspannungsphasen) und als negativ empfunden wird, wobei das Stressempfinden des einzelnen immer individuell ist \zit{StressGrundwissen} \zit{ChronischerStress}. 

Um jenen \grqq ungesunden\grqq\  Stress besser bewältigen zu können sind individuelle Erholungsphasen notwendig. Diese Phasen dienen nicht nur der Erholung  und Distanzierung von angespannten Situationen selbst, sondern auch als Präventivmaßnahme. Ist der Mensch grundsätzlich weniger angespannt, so hat er mehr Ressourcen zur Verfügung, die ihm bei der Bewältigung von stressigen Situationen helfen können. 
Beispiele für erholsame und regenerative Aktivitäten sind:
\begin{itemize}[itemsep=0.5mm, parsep=0pt]
\item  Sport
\item  Hand- und Heimwerken
\item  Spielen
\item Entspannungsübungen
\end{itemize}

\zit{Stressbewältigung}.

Mittlerweile erkennen auch viele Arbeitgeber bzw. Unternehmen, dass entspanntere Mitarbeiter meist gleichermaßen produktivere Mitarbeiter sind und bieten deshalb unter anderem Meditations- und Achtsamskeitkurse für Mananger oder teils gesamte Belegschaften an \zit{Handelsblatt}. Das Softwareunternehmen Adobe beispielsweise unterstützt seine Mitarbeiter mit verschiedenen Hilfestellungen und vergibt kostenfreie Abonnements für die Meditations- und Entspannungs-App \textit{Headspace} \zit{Adobe}. 
Aber auch außerhalb der Arbeitswelt entdecken immer mehr Menschen Apps für sich, die schnelle Entspannung für die breite Masse versprechen. So haben sich die Ausgaben für mobile Gesundheits- und Meditations-Apps zwischen 2016 und 2018 global rund verdreifacht \zit{SteigendeNutzung}.

Das Ziel dieser Arbeit ist es nun zu untersuchen, in wieweit derartige mobile Applikationen zur Stressbewältigung explizit bei Studenten beitragen können.
Neben einer qualitativen Datenerhebung mithilfe einiger Interviews der entsprechenden Zielgruppe, dient vor allem auch eine Metaanalyse bereits vorhandener Studien und Literatur als Grundlage für die spätere Aufstellung neuer Hypothesen.
\bigbreak


\section{GRUNDLAGEN UND RELATED WORKS}

\subsection{Stress}
\subsection{Mindfulness}
Mindfulness, zu Deutsch Achtsamkeit, beschreibt eine gewisse Haltung bzw. bestimmte Prinzipien, die ihren Ursprung im Buddhismus haben und heutzutage unter anderem als Grundlage für verschiedene Therapiekonzepte in der Psychotherapie dienen. Der Molekularbiologe Kabat-Zinn beschreibt Achtsamkeit als absichtlichen, geistesgegenwärtigen Zustand ohne jegliche Wertung dessen. Kabat-Zinn entwickelte außerdem das sog. \grqq Mindfulness-based Stress Reduction-Programm\grqq\ (folgend MBSR abgekürzt), welches ein achtwöchiges Trainingsprogamm beschreibt das Übungen aus dem Yoga, sowie Übungen zur eigenen Körperwahrnehmung und Aufmerksamkeit umfasst \zit{AchtsamkeitBasics}.  Ein wichtiger Bestandteil der Achtsamkeitslehre sind Meditationen und meditationsähnliche Übungen, deren positive Auswirkungen auf die physische Gesundheit durch eine Vielzahl an Studien bestätigt werden konnte \zit{MindfulnessEffects}.

\subsection{Mindfulness Apps}
Für diese Arbeit definiert die Begrifflichkeit  \grqq Mindfulness Apps\grqq\ bestimmte Applikationen für mobile Endgeräte bzw. Smartphones. Die Anwendungen vermitteln dem Benutzer verschiedene Meditationspraktiken (z.B. durch geführte Meditation) und weitere Übungen und Funktionen (z.B. Aufmerksamkeits-Übungen) um kognitive Fähigkeiten trainieren zu können. 2018 war die App \textit{Headspace} in den Ländern Kanada, England und Holland jeweils unter den Anwendungen vertreten, für die die Nutzer bereit waren am meisten Geld auszugeben \zit{SteigendeNutzung}. Deshalb wurde diese Anwendung im Rahmen dieser Arbeit analysiert und dient teilweise als Grundlage für weitere Nachforschungen und Teile der späteren Interviewfragen. 
Weitere Applikationen die in das Raster der Mindfulness Apps fallen sind beispielsweise \textit{Calm}, \textit{Relax} \textit{Meditation}, \textit{10\% Happier} und\textit{ The Mindfulness App }\zit{VergleichApps}. Die Wirksamkeit bzw. der Nutzen solcher Anwendungen wurde bereits in verschiedenen Studien überprüft (siehe Abschnitt Related Works).
\subsection{Related Works}


\section{METHODIK UND DURCHFÜHRUNG}
Um die Forschungsfrage beantworten zu können wird die Arbeit in zwei Aufgabenbereiche gegliedert, einer Metaanalyse bereits vorhandener Literatur und die Durchführung mehrerer Leitfadeninterviews. 
Eine Metaanalyse im herkömmlichen Sinn, also eine quantitative Kodierung und Auswertung vorhandener Forschung, ist aufgrund des geringen (zeitlichen) Umfangs dieser Arbeit nicht möglich. Vielmehr erfolgt dieser Arbeitsschritt in Annäherung an eine systematische Überssichtsarbeit, um relevante Forschungsarbeiten zusammenfassen und letztendlich die gegebene Fragestellung besser beantworten zu können \zit{Metaanalyse}.

Die Schlussfolgerungen, die die Auswertungen der bereits bestehenden Quellen und Related Works (siehe Abschnitt 2) zulassen, dienen als Grundlage für die Erstellung bzw. Entwicklung eines Fragebogens für Leitfadeninterviews. Leitfadeninterviews bezeichnen eine qualitative Erhebungs- bzw. Forschungsmethode bei der mehrere Interviews mithilfe eines vorher definierten, mehr oder weniger strukturierten, Fragebogens geführt werden. Für diese Arbeit wurde ein Fragenkatalog erstellt, der je nach Gesprächsverlauf flexibel abgearbeitet werden kann. Kommen während des Interviews weitere, neue Fragen / Antworten auf, sind auch diese zulässig. 
Die Auswahl der Gesprächspartner erfolgt hauptsächlich nach dem Kriterium, dass diese in der Zielgruppe, also aktuell Studierende, sein müssen.  WARUM INTERVIEW

Die eigentlichen Interviews werden als Einzelgespräche über einen Audio- bzw. Videoanruf auf den Plattformen \textit{Zoom} bzw. \textit{Discord} geführt und mit Einverständnis des jeweiligen Gegenübers aufgezeichnet. 
Im Nachgang erfolgt das Transkribieren der Gespräche in zwei Schritten. Erst wird die jeweilige Audioaufzeichnung mithilfe der computergestützten Spracheingabe auf der Plattform \textit{Google Docs} zu Text umgewandelt. Im Nachgang erfolgt ein manuelles \grqq Sichten\grqq\ der Aufnahme und gegebenenfalls das Ausbessern von fehlerhaften oder nicht transkribierten Aussagen. 
Sind alle Interviews transkribiert, erfolgt eine qualitative Inhaltsanalyse nach Kuckartz in mehreren Schritten. Zu Beginn werden die vorhandenen Transkripte ausgewertet und erste thematische Hauptgruppen generiert. Auf Grundlage dieser Gruppen werden alle vorliegenden Interviews codiert und die jeweilig gleich codierten Stellen zusammengestellt um daraus induktiv sog. Subkategorien bestimmen zu können. Sind alle Inhalte entsprechend codiert bzw. kategorisiert, kann mit der eigentlichen Auswertung begonnen werden. Hier wird versucht neue Zusammenhänge zwischen verschiedenen Kategorien und bereits bestehendem Wissen zu erkennen, um im Nachgang die Ergebnispräsentation vorbereiten zu können \zit{Inhaltsanalyse}.

\subsubsection{Durchführung}
Für die Durchführung der Interviews wurde mit jedem Befragten ein individueller Termin vereinbart, zu dem sich dann auf der vorher bestimmten Plattform getroffen wurde. Nach einer kurzen Begrüßung wurde der Befragte kurz darüber informiert, dass es sich bei dem Interview grob um das Thema "Stress" handle. Weiterhin, dass das Gespräch aufgezeichnet und später transkribiert wird, um es für die Arbeit auswerten zu können. Bevor die eigentlichen Fragen gestellt wurden, wurde das Gegenüber darüber aufgeklärt, dass die Befragung anonym verarbeitet wird, Fragen ehrlich und ausführlich beantwortet werden sollen und bei Fragen die nicht beantwortet werden möchten dies kommuniziert und keine Falschaussage getroffen werden soll. Im Anschluss an das Interview wurde sich bei der befragten Person für die aufgebrachte Zeit bedankt und gegebenenfalls weitere Informationen über die Arbeit bzw. das Ziel der Arbeit gegeben.
Insgesamt wurden 6 Interviews durchgeführt, wovon alle Gesprächspartner (drei weibliche, drei männliche Personen) aktuell eingeschriebene Studenten an einer bayrischen Hochschule oder Universität sind. 

\subsubsection{Fragenkatalog}

\section{AUSWERTUNG}
\section{FAZIT}

\nocite{*}
\printbibliography

\end{document}