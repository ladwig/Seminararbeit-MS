% Dokumentenklasse und Seitenränder
\documentclass[10pt]{article}
\usepackage[a4paper, total={6in, 8in}]{geometry}

% Font-Einstellungen: Arial
\usepackage{fontspec}
\setmainfont[
BoldFont=arialbd.ttf,
ItalicFont=ariali.ttf,
BoldItalicFont=arialbi.ttf
]{arial.ttf}

% First Indent und CSQuotes
\usepackage{indentfirst}
\usepackage{csquotes}

% \maketitle überschreiben für linksbündigen Text
\makeatletter
\renewcommand{\maketitle}{\bgroup\setlength{\parindent}{0pt}
\begin{flushleft}
  \Large{\textbf{\@title}}
\end{flushleft}\egroup
}
\makeatother

% \section überschreiben für kleinere Schriftgröße
\usepackage{titlesec}
\titleformat*{\section}{\normalsize\bfseries}

% Packages für Referenzen...
\usepackage[backend=biber,style=authoryear,citestyle=authoryear]{biblatex}
\setlength\bibitemsep{1.0\itemsep}
\addbibresource{references.bib}

% Deutsche Referenzen
\usepackage[ngerman]{babel}

% Zeilenabstand
\usepackage{setspace}
\onehalfspacing

% Globale Einrückungen je Absatz unterbinden
\setlength{\parindent}{0pt}

% Makro von StackExchange für sekundäre Quellen
\DeclareCiteCommand{\secondaryciteauthor}{}{\printtext[bibhyperref]{\printnames{labelname}}}{}{}
\DeclareCiteCommand{\secondaryciteyear}{}{\bibhyperref{\printdate}}{\multicitedelim}{}
\newcommand{\szit}[2]{(\secondaryciteauthor{#1} \secondaryciteyear{#1}, zitiert nach \secondaryciteauthor{#2} \secondaryciteyear{#2})}

% Makro für Klammern um Zitierung
\newcommand{\zit}[1]{(\cite{#1})}

% ======================
% ANFANG DOKUMENT
% ======================

\title{Stressreduzierende Wirkung von Mindfulness Smartphone-Apps bei regelmäßiger Nutzung}

\begin{document}

\maketitle

\begin{flushleft}
Daniel Ladwig \\* 
Hochschule für angewandte Wissenschaften Würzburg-Schweinfurt
\end{flushleft}



\section{EINLEITUNG}

Rund jeder vierte Studierende fühlt sich im Studienalltag ziemlich bis häufig gestresst \zit{gesundheitStudis2017}. Auch gesamtbevölkerisch ist erkennbar, dass sich vor allem junge Menschen immer gestresster fühlen. Als Auslöser werden hauptsächlich Beruf, Ausbildung bzw. Studium, aber auch soziale Verpflichtungen wie Termine und Freizeitaktivitäten genannt\zit{tkEntspannDich2016}.
\bigbreak


\section{Haupteil}
In den Vereinigten Staaten, im Zeitraum von 2012 bis 2018, hat sich der Anteil der Studenten, die an einem Fernunterricht teilnehmen, von 25,9 \% auf 35,3 \% (+9,4 \%) erhöht. Von diesen 35,3 \% nahmen 16,6 \% der Studenten ausschließlich online teil \zit{distanceLearningUSies}{distanceLearningUSstatista}. \bigbreak


\nocite{*}
\printbibliography

\end{document}
